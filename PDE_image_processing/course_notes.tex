\documentclass[11pt,a4paper,noindent]{article}
\usepackage{graphicx,amsmath,amsfonts,amssymb,amsthm}
\usepackage{fullpage}
\usepackage{exercise}
\usepackage{listings}
\usepackage{color} %red, green, blue, yellow, cyan, magenta, black, white
\definecolor{mygreen}{RGB}{28,172,0} % color values Red, Green, Blue
\definecolor{mylilas}{RGB}{170,55,241}

\begin{document}
\lstset{language=Matlab,%
    %basicstyle=\color{red},
    basicstyle=\small,
    breaklines=true,%
    morekeywords={matlab2tikz},
    keywordstyle=\color{blue},%
    morekeywords=[2]{1}, keywordstyle=[2]{\color{black}},
    identifierstyle=\color{black},%
    stringstyle=\color{mylilas},
    commentstyle=\color{mygreen},%
    showstringspaces=false,%without this there will be a symbol in the places where there is a space
    numbers=left,%
    numberstyle={\tiny \color{black}},% size of the numbers
    numbersep=9pt, % this defines how far the numbers are from the text
    emph=[1]{for,end,break},emphstyle=[1]\color{red}, %some words to emphasise
    %emph=[2]{word1,word2}, emphstyle=[2]{style},    
}

\title{PDEs for Image Processing}
\date{August 15, 2014}
\author{Martin Robinson}

\maketitle

%\section{Separation of Variables - ODE versus PDE}
%An ordinary differential equation is an equation containing a single variable and its derivatives. For example
%$$
%\frac{d}{dt} f(t) = g(t) h(f(t))
%$$
%If we define $y = f(t)$, then we can separate $x$ and $y$ to either side of the equation
%$$
%\frac{1}{h(y)}dy = g(t) dt
%$$
%If we are given the functions $h$ and $g$, then we can (possibly) integrate both sides to find a solution.
%
%We can take a simple example where $g(t) = 1$ and $h(y) = ky$
%$$
%\frac{dy}{dt} = ky \\
%$$
%This is separated into 
%$$
%\frac{1}{y} dy = k dt
%$$
%Taking the integral of both sides and using initial condition $y(0) = y_0$ gives
%$$
%y(t) = y_0 e^{kt}
%$$

\section{Image Domain}
A grey-scale image is simply a two dimensional $NxM$ array of pixel values $u_{i,j}$, where $i \in {1\ldots L_x}$ and $j \in {1\ldots L_y}$. However, we can also consider that $u_{i,j}$ are samples of some continuous function $u(x,y)$ defined on the domain $(x,y) \in \Omega \subset \mathbb{R}^+$, where $\mathbb{R}^+$ is the set of positive real numbers. For the sake of simplicity, we set the domain so that $\Omega = \{x,y \in \mathbb{R}^+: x < L_x \wedge y < L_y\}$, so that the distances between neighbouring pixels will be equal to 1.

In this case, an image consists of samples from this continuous function

\begin{align}
u_{ij} = u(\mathbf{x}_{i,j}) \\
\mathbf{x}_{i,j} = \begin{pmatrix} i-0.5 \\ j-0.5 \end{pmatrix}
\end{align}

We can consider a "real" image to consist of both a source term $u$, which we are trying to recover, and an additive noise term $n$

\begin{equation}
u_{i,j} = u(\mathbf{x}_{i,j}) + n(\mathbf{x}_{i,j})
\end{equation}

The problem then becomes the recovery of the source term $u$ from the set of pixel values $u_{i,j}$. That is, we need to construct some sort of operator to remove the noise term $n$.

\section{Signal Processing - Low-Pass Filtering}

Assuming that the noise term is comprised of mainly high frequency terms and the source term is mainly low frequency terms, one might use a low-pass filter to (approximately) recover the source term.

A low-pass filter is a filter that removes or attenuates some of the higher frequency components of the original signal. The main parameter to the low-pass filter is the \emph{cut-off} frequency, above which most of the attenuation is performed.

A low-pass filter can be implemented by convolution of the pixel values by a function $f$. In one dimension and in an infinite domain, the convolution operator is defined as

\begin{equation}
(f \star u)(x) = \int_{-\infty}^{\infty} f(x-x')u(x')dx'
\end{equation}

%The discrete convolution is given by
%
%\begin{equation}
%(f \star u)_{i} = \sum_{n=-\inf}^{\inf} f_{i-n} u_{n}
%\end{equation}

A common low-pass filter is the \emph{Gaussian} filter, which uses a gaussian function for $f$

\begin{equation}
f(x) = \frac{1}{\sqrt{2 \pi} \sigma} e^{-\frac{x^2}{2\sigma^2}}
\end{equation}

We can examine the frequency response of the gaussian filter by considering its \emph{fourier transform} ($\mathcal{F}(f) = \hat{f}$), defined as

\begin{equation}
\hat{f}(k) = \int_{-\infty}^{\infty} f(x) e^{-ikx} dx,
\end{equation}

where $k$ is the mode frequency in units of radians per unit length. Note that this integral only converges if $f(x) \rightarrow 0$ fast enough as $x \rightarrow  \pm \infty$. This transforms our original function from the spatial domain ($x$) to the frequency domain ($k$). It is equivilent to the fourier series in the limit of infinite domain size $L \rightarrow \infty$. The original function can be recovered by the inverse fourier transform

\begin{equation}
f(x) = \frac{1}{2\pi} \int_{-\infty}^{\infty} \hat{f}(x) e^{ikx} dk.
\end{equation}

\subsubsection*{Some fourier transform properties}
\begin{enumerate}
\item $\mathcal{F}$ is a linear operator: $\mathcal{F}\{af + bg\} = a\mathcal{F}\{f\} + b\mathcal{F}\{g\}$
\item Shift property: $\mathcal{F}\{f(x+a)\} = e^{ika} \hat{f}(k).$
\begin{equation}
\mathcal{F}{f(x+a)} = \int_{-\infty}^{\infty} f(x+a) e^{-ikx} dx
 				    = \int_{-\infty}^{\infty} f(x') e^{ika} e^{-ikx'} dx'
 				   = e^{ika} \hat{f}(k).
\end{equation}
\item Derivative property: $\mathcal{F}\{\frac{df}{dx}\} = ik \hat{f}(k).$
\begin{equation}
\mathcal{F}{\frac{df}{dx}} = \int_{-\infty}^{\infty} \frac{df}{dx} e^{-ikx} dx
= \left[f(x)e^{-ikx}\right]^\infty_{-\infty}+ik\int_{-\infty}^{\infty}f(x)e^{-ikx}dx
 				   = ik\hat{f}(k).
\end{equation}
\item Convolution property: $\mathcal{F}\{f(x) \star g(x)\} = \hat{f}(k)\hat{g}(k)$
\begin{align}
\mathcal{F}{f \star g} &= \int_{-\infty}^{\infty} \left [ \int_{-\infty}^{\infty} f(x-x')g(x')dx' \right ] e^{-ikx} dx \\
&= \int_{-\infty}^{\infty} g(x') e^{-ikx'} \left [ \int_{-\infty}^{\infty} f(x-x')e^{-ik(x-x')}dx \right ]  dx' \\
&= \int_{-\infty}^{\infty} g(x') e^{-ikx'} \left [ \int_{-\infty}^{\infty} f(x-x')e^{-ik\xi}d\xi \right ]  dx' \\
&= \hat{f}(k)\hat{g}(k)
\end{align}

\item Fourier transform of gaussian: $\hat{f}(k) = e^{-\frac{k^2}{2\sigma^2}}$
\begin{align}
\hat{f}(k) &= \int_{-\infty}^{\infty}  e^{-\frac{x^2}{2\sigma^2}} e^{-ikx} dx \\
\frac{d}{dk} \hat{f}(k) &=  \int_{-\infty}^{\infty}  e^{-\frac{x^2}{2\sigma^2}} (-ix) e^{-ikx} dx \\
&= i\sigma^2 \int_{-\infty}^{\infty}  \left(\frac{d}{dx} e^{-\frac{x^2}{2\sigma^2}}\right)  e^{-ikx} dx \\
&= -k\sigma^2 \int_{-\infty}^{\infty}  e^{-\frac{x^2}{2\sigma^2}} e^{-ikx} dx \\
&= -k\sigma^2 \hat{f}(k)
\end{align}
Therefore
\begin{equation}
\hat{f}(k) = C e^{-\frac{k^2}{2\sigma^2}}
\end{equation}
where the constant $C$ is determined using $C = \hat{f}(0) = \int_{-\infty}^{\infty} \frac{1}{\sqrt{2\pi} \sigma} e^{-\frac{x^2}{2\sigma^2}} dx$ so that $C = 1$
\end{enumerate}

Using the convolution property and the fourier transform of a guassian, it can be seen that the application of a gaussian filter attenuates the frequency components of the original image by a gaussian. i.e. a low-pass filter

\section{Heat Equation}

Another option is to remove the noise term by applying a numerical discritisation of the \emph{diffusion equation}. As we will see, this is equivalent to the gaussian filter, but has the advantage that we can easily modify the equation to, for example, recover edges in the image.

The diffusion, or heat equation in one-dimension is given by

\begin{equation}\label{eq:diffusion}
\left\lbrace \begin{array}{c}
u_t(x,t) = D u_{xx}(x,t) \\
u(x,0) = u_0(x)
\end{array}
\right.
\end{equation}

where $D$ is the diffusion constant, $u_t(x,t)=\frac{\partial}{\partial t} u(x,t)$ and $u_{xx}(x,t) =  \frac{\partial^2}{\partial^2 x} u(x,t)$. You can see that the speed of evolution $u_t$ is proportional to the concavity $u_{xx}$. In other words, small features with high $u_{xx}$ (i.e. noise) will be smoothed out first.

\subsection{Fundamental Solution}

We can consider the heat equation on an infinite domain $-\infty < x < \infty$ where the function $u$ decays to zero towards infinity: $u(x,t) \rightarrow 0$ as $x \rightarrow \pm \infty$

Take the fourier transform of both sides
\begin{align}
\mathcal{F}\{u_t(x,t)\} &= \mathcal{F}\{D u_{xx}(x,t)\} \\
\frac{\partial}{\partial t} \int_{-\infty}^{\infty} f(x) e^{-ikx} dx &= D (ik) \hat{u_x} \\
 \int_{-\infty}^{\infty} \frac{\partial}{\partial t}f(x) e^{-ikx} dx &= D (ik)^2 \hat{u} \\
\hat{u}_t &= -D k^2 \hat{u}, \\
\end{align}
using the derivative property of the fourier transform and the fact that partial derivative of $t$ does not interfere with the transform. Using this result, we can see that the solution for $\hat{u}(k,t)$ is
\begin{equation}
\hat{u}(k) = C(k) e^{-k^2Dt},
\end{equation}
where the constant $C$ can be determined from the fourier transform of the initial conditions
\begin{equation}
\hat{u}(k,0) = \hat{u}_0(k) = C(k)
\end{equation}
giving
\begin{equation}
\hat{u}(k,t) = \hat{u}_0(k) e^{-k^2Dt},
\end{equation}
This can be written as
\begin{equation}
\hat{u}(k,t) = \hat{u}_0(k) \mathcal{F} \left( \frac{e^{-x^2/4Dt}}{\sqrt{4\pi Dt}} \right)
\end{equation}
Using the convolution property of the fourier transform, we can see that the inverse fourier transform of both sides gives

\begin{equation}
u(x,t) = \frac{1}{\sqrt{4 \pi D t}}  \int_{-\infty}^{\infty} u_0(x') e^{-\frac{(x-x')^2}{4Dt}} dx'
\end{equation}
where
\begin{equation}
G(x,t) = \frac{1}{\sqrt{4 \pi D t}}  e^{-\frac{x^2}{4Dt}}
\end{equation}
is the Green function, or fundamental solution, to the heat equation. Note that this is the same as the gaussian low-pass filter with $\sigma = \sqrt{2Dt}$.

%We want to find a solution to the heat equation that is the convolution of some function $G(x,y)$ with the initial condition $u_0(x)$
%
%\begin{equation} \label{eq:fundsol}
%u(x,t) = \int_{-\infty}^{\infty} G(x-y,t)u_0(y)dy
%\end{equation}
%
%Substitute Eq. \ref{eq:fundsol} back into Eq. \ref{eq:diffusion}
%
%\begin{equation}
%u_t(x,t) - D u_{xx}(x,t) = \int_{-\infty}^{\infty} (G_t(x-y,t) - D G_{xx}(x-y,t)) u_0(y) dy
%\end{equation}
%
%So if $G(x,t)$ is a solution of the heat equation, then we can see that Eq. \ref{eq:fundsol} will be as well.
%
%To find $G(x,t)$, we need to ensure that the solution $u(x,t)$ approaches the initial condition as $t\rightarrow 0^+$, that is $\lim_{t\rightarrow 0^+} u(x,t) = u_0(x)$. It turns out that
%
%\begin{equation}
%G(x,t) = \frac{1}{\sqrt{4 \pi D t}}e^{-\frac{x^2}{4kt}}
%\end{equation}
%
%This is the Green function, or fundamental solution, to the heat equation. Note that this is the same as the gaussian low-pass filter with $\omega = \sqrt{2Dt}$.
%
%\begin{Exercise}
%Verify that $G(x,t)$ is a solution of the heat equation and show informally that $\lim_{t\rightarrow 0^+} \int_{-\infty}^{\infty} G(x-y,t)u_0(y)dy = u_0(x)$
%\end{Exercise}

\subsection{Finite Differences}
\subsubsection{Forward Time, Central Space}
We will approximate the solution to the heat equation using finite differences, using a forward time, central space method.

First we calculate the time derivative using a forward difference
\begin{equation}
u_t(x_{i},t_n) \approx \frac{u^{n+1}_i-u^{n}_i}{\Delta t}
\end{equation}
where $t_n = \Delta t n$ for the set of positive integers $n = 0,\ldots, N$ and $x_i = i-0.5$ for the set of positive integers $i = 1, \ldots, L$. Note that this gives a simple grid spacing $\Delta x = 1$.

We approximate $u_{xx}$ using a central difference
\begin{equation}
u_{xx}(x_i,t_n) \approx \frac{1}{{\Delta x}^2} (u^n_{i+1} - 2u^n_{i} + u^n_{i-1}) = u^n_{i+1} - 2u^n_{i} + u^n_{i-1}
\end{equation}

Putting these into the heat equation gives
\begin{align} \label{eq:finitedifference}
\frac{u^{n+1}_i-u^{n}_i}{\Delta t} &= D (u^n_{i+1} - 2u^n_{i} + u^n_{i-1}) \\
u^{n+1}_i &= u^{n}_i + \Delta t D (u^n_{i+1} - 2u^n_{i} + u^n_{i-1}) \\
u^{n+1}_i &= \Delta t D u^n_{i+1} + (1-2\Delta t D)u^n_{i} + \Delta t D u^n_{i-1}
\end{align}

\subsubsection{Boundary Conditions}
We set our domain for $x$ to be $0 < x < L$, and set boundary conditions at $x=0$ and $x=L$.

A Dirichlet boundary condition gives the value of the function $u(x)$ at the boundary. For image processing, we might want to set a dirichlet boundary condition as the initial pixel values at the edges.
\begin{align}
u(0,t) = u_0(0)\\
u(L,t) = u_0(L).
\end{align}

A Neumann boundary condition gives the derivative $\partial / \partial x$ of the function $u(x)$ at the boundary. A zero-Neumann b.c. is often used to ensure that the gradient of the image pixels at the boundary is zero. 

\begin{align}
\frac{\partial u}{\partial x}(0)=0\\
\frac{\partial u}{\partial x}(L)=0
\end{align}

This could be implemented, for example, by having a set of "mirror" pixels surrounding the image.

\begin{align}
u_{0,j} &= u_{1,j}\\
u_{L+1,j} &= u_{L,j}
\end{align}


\subsubsection{Matlab Code}
We can write Matlab code to calculate this, using a zero-neumann boundary condition.

\begin{lstlisting}
function u = isotropic_diffusion(u, dtD, N)

  % Get size of initial condition matrix
  nx = size(u);

  % Setup indexing. We don't update the boundary pixels, as they are set by
  % the boundary conditions
  j=2:nx-1;
  
  % Loop through timesteps
  for it=1:N
      % Use Forward Time, Centered Space
      u(i,j)= dtD*(u(i+1)+u(i-1)) + (1-2*dtD)*u(i);
      
      % Zero-neumann boundary conditions
      u(1) = u(2);
      u(nx) = u(nx-1);
  end
end
\end{lstlisting}

%\subsection{Error}
%The error of the finite difference approximation is given by
%\begin{equation} \label{eq:error}
%E_i^n = \frac{u(x_i,t^{n+1})-u(x_i,t^{n})}{\Delta t} - D (u(x_{i+1},t^{n}) - 2u(x_{i1},t^{n}) + u(x_{i-1},t^{n})
%\end{equation}
%We then use the taylor expansion around $u_i^n$
%\begin{align}
%u(x_i,t_{n+1}) &= u(x_i,t_{n}) + \Delta t u_t(x_i,t_{n}) + \frac{1}{2} \Delta t^2 u_{tt}(x_i,t_{n}) + \ldots \\
%u(x_{i+1},t_{n}) &= u(x_i,t_{n}) + u_x(x_i,t_n) + \frac{1}{2} u_{xx}(x_i,t_{n}) + \ldots
%\end{align}
%Substituting this back into Eq. \ref{eq:error} gives
%\begin{equation}
%E_i^n = \frac{1}{2} \Delta t u_{tt}(x_i,t_{n}) + \mathcal{O}(\Delta t^2) - \frac{1}{12} h^2 u_{xxxx}(x_i,t_{n}) + \mathcal{O}(\Delta h^4)
%\end{equation}

\subsubsection{Stability}
The error of the finite difference approximation is given by
\begin{equation} \label{eq:error}
E_i^n = \frac{u(x_i,t^{n+1})-u(x_i,t^{n})}{\Delta t} - D (u(x_{i+1},t^{n}) - 2u(x_{i1},t^{n}) + u(x_{i-1},t^{n})
\end{equation}
Let $e_i^n = u_i^n - u(x_i,t_{n})$, then subtracting Eq. \ref{eq:error} from Eq. \ref{eq:finitedifference} gives
\begin{equation}
E_i^n = \frac{e^{n+1}_i-e^{n}_i}{\Delta t} - D (e^n_{i+1} - 2e^n_{i} + e^n_{i-1})
\end{equation}
and
\begin{equation}
e^{n+1}_i = r e^n_{i+1} + (1-2r)e^n_{i} + r e^n_{i-1} - \Delta t E_i^n
\end{equation}
let $e^n = max_i|e_i^n|$ and assume $1-2r\geq 0$
\begin{equation}
|e_i^{n+1}| \leq (1-2r)e^n + re^n + re^n + \Delta t |E_i^n|
\end{equation}
or
\begin{equation}
|e_i^{n+1}| \leq e^n + \Delta t |E_i^n|
\end{equation}
We assume that $e_0=0$, and with $E = \max_{i,n}|E_i^n|$
\begin{align}
|e_i^{n+1}| &\leq e^n + \Delta t E \geq e^{n-1} + 2\Delta t E \geq e^{n-2} + 3\Delta t E \geq \ldots \\
|e_i^{n+1}| &\leq (n+1)\Delta t E = t_{n+1} E
\end{align}
We know that the approximation is first order in time, therefore $E \leq C \Delta t$ and
\begin{equation}
|e_i^{n+1}| \leq t_{n+1}\Delta t C
\end{equation}
Take the limit as $n \rightarrow \infty$, $\Delta t \rightarrow 0$ and $t_n \rightarrow t$
\begin{equation}
|e_i^{n+1}| \leq t E \rightarrow 0
\end{equation}
Therefore the scheme converges for $1-2r\geq 0$

\subsection{2D Heat Equation}
The two-dimensional version of the heat equation is given as
\begin{equation}\label{eq:diffusion2}
\left\lbrace \begin{array}{c}
u_t(x,y,t) = D (u_{xx}(x,y,t) + u_{xx}(x,y,t)) \\
u(x,y,0) = u_0(x,y)
\end{array}
\right.
\end{equation}

\subsubsection{Finite Differences: Forward Time, Central Space}
We will approximate the solution to the heat equation using finite differences, using a forward time, central space method.

The time derivative is unchanged from 1 dimension:
\begin{equation}
u_t(x_{i},y_{j},t_n) \approx \frac{u^{n+1}_{i,j}-u^{n}_{i,j}}{\Delta t}
\end{equation}
where $t_n = \Delta t n$ for the set of positive integers $n = 0,\ldots, N$, $x_i = i-0.5$ for the set of positive integers $i = 1, \ldots, L$ and $y_j = j-0.5$ for the set of positive integers $j = 1, \ldots, L$.

We approximate $u_{xx}$ and $u_{yy}$ using a central difference
\begin{align}
u_{xx}(x_{i},y_{j},t_n) &\approx \frac{1}{{\Delta x}^2} (u^n_{i+1,j} - 2u^n_{i,j} + u^n_{i-1,j}) = u^n_{i+1,j} - 2u^n_{i,j} + u^n_{i-1,j} \\
u_{yy}(x_{i},y_{j},t_n) &\approx \frac{1}{{\Delta y}^2} (u^n_{i,j+1} - 2u^n_{i,j} + u^n_{i,j-1}) = u^n_{i,j+1} - 2u^n_{i,j} + u^n_{i,j-1} \\
\end{align}

Putting these into the heat equation gives
\begin{align} \label{eq:finitedifference}
\frac{u^{n+1}_{i,j}-u^{n}_{i,j}}{\Delta t} &= D (u^n_{i+1,j} + u^n_{i-1,j} + u^n_{i,j+1} + u^n_{i,j-1} - 4u^n_{i,j}) \\
u^{n+1}_i &= u^{n}_i + \Delta t D (u^n_{i+1,j} + u^n_{i-1,j} + u^n_{i,j+1} + u^n_{i,j-1} - 4u^n_{i,j}) \\
u^{n+1}_i &= \Delta t D (u^n_{i+1,j} + u^n_{i-1,j} + u^n_{i,j+1} + u^n_{i,j-1}) + (1-4\Delta t D)u^n_{i,j}
\end{align}

Heat equation is linear, so the same arguments for accuracy and stability in 1D also apply for 2D.

\end{document}